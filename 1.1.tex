\documentclass{article}
\usepackage{amsfonts}
\usepackage{amstext}
\usepackage{subcaption}
\usepackage{pgfplots}
\pgfplotsset{compat = newest}

\title{Even solutions for Richard H. Hammack's Book of Proof}
\author{Martin Jaskulla}
\begin{document}
\maketitle
\section{Chapter}
\subsection{Section}

\begin{enumerate}
    \item [2.] $\{3x+2:x \in \mathbb{Z} \} = \{...,-4,-1,2,5,8,...\}$
    \item [4.] $\{x \in \mathbb{N}: -2 < x \leq 7 \} =\{1,2,3,4,5,6,7\}$
    \item [6.] $\{x \in \mathbb{R}: x^2=9 \} =\{-3,3\}$
    \item [8.] $\{x \in \mathbb{R}:x^3+5x^2=-6x \} =\{0,-2,-3\}$
    \item [10.] $\{x \in \mathbb{R}: cos x = 1 \} =\{...,-2\pi,0,2\pi,...\}$
    \item [12.] $\{x \in \mathbb{Z}: |2x|<5 \} =\{-2,-1,0,1,2\}$
    \item [14.] $\{5x: x \in \mathbb{Z}, |2x| \leq8 \} =\{-20,-15,-10,-5,0,5,10,15,20\}$
    \item [16.] $\{6a+2b: a,b \in \mathbb{Z} \} =\{...,-4,-2,0,2,4,...\}$
    \item [18.] $\{0,4,16,36,64,100,...\} = \{x^2: x \in \mathbb{W}, \text{x is even}\}$
    \item [20.] $\{...,-8,-3,2,7,12,17,...\} = \{5x+2: x \in \mathbb{Z}\}$
    \item [22.] $\{3,6,11,18,27,38,...\} = \{x^2+2:x\in \mathbb{Z}\}$
    \item [24.] $\{-4,-3,-2,-1,0,1,2\} = \{ x: x\in \mathbb{Z}, -4 \leq x \leq 2\}$
    \item [26.] $\{...,\frac{1}{27},\frac{1}{9},\frac{1}{3},1,3,9,27,...\} = \{ 3^x: x \in \mathbb{Z}\}$
    \item [28.] $\{...,-\frac{3}{2},-\frac{3}{4},0,\frac{3}{4},\frac{3}{2},\frac{9}{4},3,\frac{15}{4},\frac{9}{2},...\} =\{x * \frac{3}{4}:x \in \mathbb{Z}\}$
    \item [30.] $|\{\{1,4\},a,b,\{\{3,4\}\},\{\emptyset\}\}| = 5$
    \item [32.] $|\{\{\{1,4\},a,b,\{\{3,4\}\},\{\emptyset\}\}\}| = 1$
    \item [34.] $|\{ x \in \mathbb{N}: |x|<10 \}| = 9$
    \item [36.] $|\{ x \in \mathbb{N}: x^2<10 \}| = 3$
    \item [38.] $|\{ x \in \mathbb{N}: 5x \le 20 \}| = 4$
\end{enumerate}

\clearpage
\begin{figure}
    \centering
    \begin{subfigure}[b]{0.49\textwidth}
        \centering
        \caption*{40. $\{(x,y):x \in [0,1], y\in [1,2] \}$}
        \resizebox{1\linewidth}{!}{\input{./python/graphs/1.1.40.pgf}}
    \end{subfigure}
    \hfill
    \begin{subfigure}[b]{0.49\textwidth}
        \centering
        \caption*{42. $\{(x,y):x = 2,y \in [0,1] \}$}
        \resizebox{1\linewidth}{!}{\input{./python/graphs/1.1.42.pgf}}
    \end{subfigure}
    \hfill
    \begin{subfigure}[b]{0.49\textwidth}
        \centering
        \caption*{44. $\{(x,x^2):x \in \mathbb{R} \} $}
        \resizebox{1\linewidth}{!}{\input{./python/graphs/1.1.44.pgf}}
    \end{subfigure}
    \hfill
    \begin{subfigure}[b]{0.49\textwidth}
        \centering
        \caption*{46. $\{(x,y):x,y \in \mathbb{R},x^2+y^2\leq1 \}$}
        \resizebox{1\linewidth}{!}{\input{./python/graphs/1.1.46.pgf}}
    \end{subfigure}
    \hfill
    \begin{subfigure}[b]{0.49\textwidth}
        \centering
        \caption*{48. $\{(x,y):x,y \in \mathbb{R},x > 1 \}$}
        \resizebox{1\linewidth}{!}{\input{./python/graphs/1.1.48.pgf}}
    \end{subfigure}
    \hfill
    \begin{subfigure}[b]{0.49\textwidth}
        \centering
        \caption*{50. $\{(x,\frac{x^2}{y}):x \in \mathbb{R},y \in \mathbb{N} \}$}
        \resizebox{1\linewidth}{!}{\input{./python/graphs/1.1.50.pgf}}
    \end{subfigure}
\end{figure}
\clearpage
\begin{figure}
    \centering
    \begin{subfigure}[b]{0.49\textwidth}
        \centering
        \caption*{52. $\{(x,y) \in \mathbb{R}^2: (y-x^2)(y+x^2) =0 \}$}
        \resizebox{1\linewidth}{!}{\input{./python/graphs/1.1.52.pgf}}
    \end{subfigure}
\end{figure}

\subsection{Section}

\begin{enumerate}
    \item [2.] $A=\{\pi, e, 0\}, B=\{0,1\}$
    \begin{enumerate}
        \item[a)] $ A \times B = \{(\pi, 0),(\pi, 1),(e, 0),(e, 1),(0, 0),(0, 1)\}$
        \item[b)] $ B \times A = \{(0, \pi),(0, e),(0, 0),(1, \pi),(1, e),(1, 0)\}$
        \item[c)] $ A \times A = \{(\pi, \pi),(\pi, e),(\pi, 0),(e, \pi),(e, e),(e, 0),(0, \pi),(0, e),(0, 0)\}$
        \item[d)] $ B \times B = \{(0, 0),(0, 1),(1, 0),(1, 1)\}$
        \item[e)] $ A \times \emptyset = \emptyset$
        \item[f)] $ (A \times B) \times B = $
        \item[] $ \{((\pi, 0), 0),((\pi, 0), 1),((\pi, 1), 0),((\pi, 1), 1),((e, 0), 0),((e, 0), 1),$
        \item[] $ ((e, 1), 0),((e, 1), 1),((0, 0), 0),((0, 0), 1),((0, 1), 0),((0, 1), 1)\} $
        \item[g)] $ A \times (B \times B) = $
        \item[] $ \{(\pi, (0, 0)),(\pi, (0, 1)),(\pi, (1, 0)),(\pi, (1, 1)),(e, (0, 0)),(e, (0, 1)),$
        \item[] $ (e, (1, 0)),(e, (1, 1)),(0, (0, 0)),(0, (0, 1)),(0, (1, 0)),(0, (1, 1))\}$
        \item[h)] $ A \times B \times B = $
        \item[] $ \{(\pi, 0, 0),(\pi, 0, 1),(\pi, 1, 0),(\pi, 1, 1),(e, 0, 0),(e, 0, 1),$
        \item[] $ (e, 1, 0),(e, 1, 1),(0, 0, 0),(0, 0, 1),(0, 1, 0),(0, 1, 1)\}$
    \end{enumerate}
    \item [4.] $\{n \in \mathbb{Z}: 2<n<5\} \times \{n \in \mathbb{Z}:|n|=5\}=\{(3, 5),(3, -5),(4, 5),(4, -5)\}$
    \item [6.] $\{x \in \mathbb{R}: x^2=x\} \times \{x \in \mathbb{N}:x^2=x\}=\{(0, 1),(1, 1)\}$
    \item [8.] $\{0,1\}^4=\{(0, 0, 0, 0),(0, 0, 0, 1),(0, 0, 1, 0),(0, 0, 1, 1),(0, 1, 0, 0),(0, 1, 0, 1),(0, 1, 1, 0),$
    \item [] $ (0, 1, 1, 1),(1, 0, 0, 0),(1, 0, 0, 1),(1, 0, 1, 0),(1, 0, 1, 1),(1, 1, 0, 0),(1, 1, 0, 1),(1, 1, 1, 0),(1, 1, 1, 1)\}$
\end{enumerate}
\clearpage
\begin{figure}
    \centering
    \begin{subfigure}[b]{0.49\textwidth}
        \centering
        \caption*{10. $\{-1,0,1\} \times \{1,2,3\}$}
        \resizebox{1\linewidth}{!}{\input{./python/graphs/1.2.10.pgf}}
    \end{subfigure}
    \hfill
    \begin{subfigure}[b]{0.49\textwidth}
        \centering
        \caption*{12. $[-1,1] \times [1,2]$}
        \resizebox{1\linewidth}{!}{\input{./python/graphs/1.2.12.pgf}}
    \end{subfigure}
    \hfill
    \begin{subfigure}[b]{0.49\textwidth}
        \centering
        \caption*{14. $[1,2] \times \{1,1.5,2\}$}
        \resizebox{1\linewidth}{!}{\input{./python/graphs/1.2.14.pgf}}
    \end{subfigure}
    \hfill
    \begin{subfigure}[b]{0.49\textwidth}
        \centering
        \caption*{16. $[0,1]\times\{1\}$}
        \resizebox{1\linewidth}{!}{\input{./python/graphs/1.2.16.pgf}}
    \end{subfigure}
    \hfill
    \begin{subfigure}[b]{0.49\textwidth}
        \centering
        \caption*{18. $\mathbb{Z} \times \mathbb{Z}$}
        \resizebox{1\linewidth}{!}{\input{./python/graphs/1.2.18.pgf}}
    \end{subfigure}
    \hfill
    \begin{subfigure}[b]{0.49\textwidth}
        \centering
        \caption*{20. $\{(x,y):\in\mathbb{R}^2:x^2+y^2\leq1\}\times [0,1]$}
        \resizebox{1\linewidth}{!}{\input{./python/graphs/1.2.20.pgf}}
    \end{subfigure}
\end{figure}
\clearpage

\subsection{Section}
\begin{enumerate}
    \item[2] $ \mathcal{P}(\{1,2,\emptyset\}) =\{\{\},\{1\},\{2\},\{\emptyset\},\{1, 2\},\{1, \emptyset\},\{2, \emptyset\},\{1, 2, \emptyset\}\}$
    \item[4] $ \mathcal{P}(\emptyset)=\{\emptyset\}$
    \item[6] $ \mathcal{P}(\{\mathbb{R},\mathbb{Q},\mathbb{N}\}) = \{\{\},\{\mathbb{R}\},\{\mathbb{Q}\},\{\mathbb{N}\},\{\mathbb{R}, \mathbb{Q}\},\{\mathbb{R}, \mathbb{N}\},\{\mathbb{Q}, \mathbb{N}\},\{\mathbb{R}, \mathbb{Q}, \mathbb{N}\}\}$
    \item[8] $ \mathcal{P}(\{\{0,1\},\{0,1,\{2\}\},\{0\}\}) = \{\{\},\{\{0, 1\}\},\{\{0, 1, \{2\}\}\},\{\{0\}\},\\\{\{0, 1\}, \{0, 1, \{2\}\}\},\{\{0, 1\}, \{0\}\},\{\{0, 1, \{2\}\}, \{0\}\},\{\{0, 1\}, \{0, 1, \{2\}\}, \{0\}\}\}$
    \item[10] $\{X \subseteq \mathbb{N}: |X| \leq 1\} = \{\emptyset, \{1\},\{2\},\{3\},...\}$
    \item[12] $\{X: X \subseteq \{3,2,a\} \text{ and } |X|=1 \} = \{\emptyset, \{3\}, \{2\},\{a\}\}$
    \item[14] $ \mathbb{R}^2 \subseteq \mathbb{R}^3$ is false, because the point $(1,1)$ of the 2-dimensional plane $\mathbb{R}^2$ is not part of the 3-dimensional space $\mathbb{R}^3$, which has points such as $(1,1,1)$ 
    \item[16] $ \{(x,y) \in \mathbb{R}^2:x^2-x=0\} \subseteq \{(x,y) \in \mathbb{R}^2:x-1=0\} $ is false, because the set on the right does not contain point $(-1,y)$
\end{enumerate}

\subsection{Section}
\begin{enumerate}
    \item[2] $ \mathcal{P}(\{1,2,3,4\}) = \{\{\},\{1\},\{2\},\{1, 2\},\{3\},\{1, 3\},\{2, 3\},\{1, 2, 3\},\{4\},\{1, 4\},\{2, 4\},\\\{1, 2, 4\},\{3, 4\},\{1, 3, 4\},\{2, 3, 4\},\{1, 2, 3, 4\}\}$
    \item[4] $ \mathcal{P}(\{ \mathbb{R}, \mathbb{Q} \}) = \{\{\}, \{\mathbb{R}\},\{\mathbb{Q}\},\{\mathbb{R}, \mathbb{Q}\}\}$
    \item[6] $ \mathcal{P}(\{ 1,2 \}) \times \mathcal{P}(\{ 3 \})= \{(\{\}, \{\}),(\{\}, \{3\}),(\{1\}, \{\}),(\{1\}, \{3\}),(\{2\}, \{\}),(\{2\}, \{3\}),\\(\{1, 2\}, \{\}),(\{1, 2\}, \{3\}\}\}$
    \item[8] $ \mathcal{P}(\{ 1,2 \} \times \{ 3 \})= \{\{\},\{(1, 3)\},\{(2, 3)\},\{(1, 3), (2, 3)\}$
    \item[10] $ \{ X \in \mathcal{P}( \{ 1,2,3 \} ): |X| \leq 1 \}=\{\{\}, \{1\}, \{2\},\{3\} \}$
    \item[12] $ \{ X \in \mathcal{P}( \{ 1,2,3 \} ): 2 \in X \}= \{\{2\},\{1.2\},\{2,3\},\{1,2,3\}\}$
    \item[] Suppose $|A|=m, |B|=n$
    \item[14] $ | \mathcal{P}(\mathcal{P}(A)) | = 2^{2^m} $
    \item[16] $ | \mathcal{P}(A) \times \mathcal{P}(B) | = 2^m * 2^n$
    \item[18] $ | \mathcal{P}( A \times \mathcal{P}(B) ) | = 2^{m * 2^n} $
    \item[20] $ | \{ X \subseteq \mathcal{P}(A): |X| \leq 1 \} | = m + 2 $
\end{enumerate}

\subsection{Section}
\begin{enumerate}
    \item[10] The statement $ (\mathbb{R} - \mathbb{Q}) \times \mathbb{N} = (\mathbb{R} \times \mathbb{N}) - (\mathbb{Z} \times \mathbb{N})$ is true, because each point has a y value $\in \mathbb{N}$ and both side remove all $\mathbb{Z}$ values from x.
\end{enumerate}


\end{document}